% Preamble
\documentclass[11pt]{article}

% Packages
\usepackage[letterpaper,margin=1in]{geometry}
\usepackage{amsmath}
\usepackage{amsthm}
\usepackage{amsfonts}
\usepackage{amssymb}
\usepackage{fancyhdr}
\usepackage{mathtools}
\usepackage{tikz-cd}
\usepackage{stmaryrd}
\usepackage{enumerate}
\usepackage{capt-of}
\usepackage{xurl}
\usepackage{hyperref}
\usepackage{framed}
\usepackage{pgfplots}

\pgfplotsset{compat=1.18}

\pgfmathdeclarefunction{gauss}{2}{%
    \pgfmathparse{1/(#2*sqrt(2*pi))*exp(-((x-#1)^2)/(2*#2^2))}%
}

% Document
\begin{document}
    \setlength{\parindent}{0pt}
    \setlength{\parskip}{5pt}
    \setlength{\headheight}{14.49999pt}
    \addtolength{\topmargin}{-1.59999pt}

    \title{Simulating Voting Systems}
    \author{\normalsize{Russell Emerine (\url{remerine@andrew.cmu.edu})}}
    \date{}

    \maketitle

    \begin{abstract}
        Much of the past work on voting systems focuses on \textit{ranked voting systems},
        which have a number of limitations such as Arrow's Theorem~\cite{Arrow}.
        In this paper we consider ranked voting systems as well as the less commonly
        used class of \textit{rated voting systems}.
        The systems differ in that ranked voting systems only allow the voter to order the candidates,
        while in rated voting systems the voter can score each candidate independently.
        In 2000, Warren Smith \cite{Smith} evaluated ranked and rated voting systems
        under a Monte Carlo simulation model of voter utilities and behaviors.
        We replicate Smith's results with a wider selection of voting systems and voter utility distributions
        and conclude that range voting,
        a rated system where each candidate can be independently scored in the range [0, 1],
        has the best performance.

        This project's source code is freely available online at \url{https://github.com/RussellEmerine/voting_simulation}.
    \end{abstract}


    \section{Voting Systems}\label{sec:voting-systems}

    A \textit{voting system} is a system where
    a (usually large) number of \textit{voters} cast \textit{votes} of some kind,
    and the votes are used in some process to select one of
    a (usually small, but greater than one) number of \textit{candidates}.

    Some formulations allow the output to be a set of candidates or an ordering of candidates.
    For simplicity, we will only consider systems that output a single winner.
    We will also focus on cases with three or more candidates, as with two candidates,
    all reasonable voting systems exhibit reasonable behavior.

    We first describe several types of voting systems,
    then list and describe the particular voting systems we consider in our simulation.

    \subsection{Ranked and Rated Voting Systems}\label{subsec:ranked-and-rated-voting-systems}

    \subsubsection{Ranked Voting Systems}

    In a ranked voting system, vote information can be encoded into an ordering of the candidates.
    For instance, plurality voting uses the most favored candidate from each ordering,
    while Borda voting uses the whole ordering.
    Ranked voting systems traditionally output an ordering of candidates
    (which in this section we refer to as the ``outcome'');
    to choose our single winner we can simply choose the most favored candidate in the outcome.

    Arrow's theorem states that, if there are at least three candidates,
    no ranked voting system may satisfy all of the following properties~\cite{Arrow}:

    \begin{itemize}
        \item Pareto efficiency:
        If every voter places $A$ before $B$, then the outcome also places $A$ before $B$.
        (This can be weakened to non-imposition,
        i.e.\ that for any $A$ and $B$, there is some cast of votes such that
        the output places $A$ before $B$~\cite{Wilson}.)
        \item Non-dictatorship:
        There is no voter whose ordering is always the same as the outcome.
        \item Independence of irrelevant alternatives:
        In two elections with the same number of voters where each pair of
        corresponding voters has the same relative ordering of $A$ and $B$,
        the outcomes of both elections have the same relative ordering of $A$ and $B$.
    \end{itemize}

    The fact that it is impossible to have all three of these very reasonable conditions is a
    severe limitation of ranked voting systems.

    \subsubsection{Rated Voting Systems}

    Votes need not be restricted to orderings of candidates.
    Rated voting systems allow a voter to provide independent scores for each candidate,
    which can be more informative than an ordering of the candidates.
    For instance, range voting allows a voter to give each candidate their own score in the range $[0, 1]$.
    thereby allowing a voter to express how much they prefer their most favored candidate
    over their second most favored candidate.

    Since these are not ranked voting systems, Arrow's theorem does not apply,
    and it is completely possible to have Pareto efficiency,
    non-dictatorship, and independence of irrelevant alternatives.
    However, they are still subject to some restrictions, notably Gibbard's theorem,
    that no deterministic process of collective decision may satisfy all of the following properties~\cite{Gibbard}:

    \begin{itemize}
        \item The process has more than two possible outcomes.
        \item There is no voter who singlehandedly determines the outcome.
        \item The game-theoretically optimal vote for a voter will not depend on
        the voter's beliefs of what other voters will vote.
    \end{itemize}

    When applied to voting systems, this implies that non-dictatorial voting systems
    with three or more possible outcomes (ranked, rated, or otherwise)
    require some kind of strategic voting that is not completely honest.

    \begin{leftbar}
        \textit{An aside about determinism:}

        Most voting systems do not have a good way of handling ties without
        using randomness or allowing multiple winners.
        However, when the number of voters is reasonably large, the chance of a tie is negligible.
        We will consider voting processes that are deterministic when there are no ties ``good enough.''
    \end{leftbar}

    \subsubsection{Other Voting Systems}

    There are a few ``obviously bad'' voting systems that we consider,
    such as ``random winner'' and ``worst candidate''.
    These are only useful as a frame of reference,
    and are not expected to have any of the properties discussed for ranked and rated systems.

    \subsection{COAF and Non-COAF Systems}\label{subsec:coaf-and-non-coaf-systems}

    \subsubsection{COAF Systems}

    Smith's specification of compact set based, one-vote, additive, fair voting systems
    describes many common ranked and rated voting systems~\cite{Smith}.
    In a COAF system with $C$ candidates there is a compact set $S \subseteq \mathbb{R}^C$ of allowed votes,
    where $S$ is symmetric across permutations of candidates (fair).
    Each voter chooses one vote in $S$ to submit.
    Then, the votes are added, and the candidate with the greatest sum
    (or equivalently, average, referred to as ``score'' for the following proof)
    is selected as the winner.
    For instance, plurality voting is when $S = \{(1, 0, 0, \dots), (0, 1, 0, \dots), (0, 0, 1, \dots), \dots\}$,
    and range voting is when $S = [0, 1]^C$.

    Smith provides a formal proof of the optimality of the ``moving average'' strategy.
    This strategy generates the game theoretically optimal vote in any COAF system
    when given poll data in the form of a predicted ordering of candidates
    by likelihood to win the election
    (assumed to be produced by a random sample of honest pollees).
    Consider a COAF system with vote set $S$
    and without loss of generality label the candidates as $c_1, c_2, \dots c_C$ in poll order.
    Let $U_1, U_2, \dots U_C$ be the utilities of the voter for each candidate,
    under the utility model discussed later.
    The vote is generated as follows:

    \begin{itemize}
        \item Let the set $X_0$ representing the set of potential votes start as $X = S$.
        \item If $U_1 > U_2$, let $X_1$ be the subset of $X_0$ that maximizes the $1$st component.
        Otherwise, let $X_1$ to the subset of $X_0$ that minimizes the $1$st component.
        \item For each candidate $c_i$ in poll order starting from $c_2$,
        if $U_i > \sum_{j = 1}^{i - 1} U_j$, let $X_i$ be the subset of $X_{i - 1}$ that maximizes the $i$th component.
        Otherwise, let $X_i$ be the subset of $X_{i - 1}$ that minimizes the $i$th component.
    \end{itemize}

    The final set $X_C$ will consist of exactly one vector, which will be the vote.

    We present an alternate interpretation of Smith's proof.

    Let us say there are $C$ candidates, $V$ votes, and $P$ pollees,
    where $V$ and $P$ are reasonably large.
    Let us say that across all votes,
    the $i$th component of the vote vector
    lies in a distribution with mean $\mu_i$ and variance $\sigma_i^2$.
    $\mu_i$ is the actual score the candidate will receive.
    $\sigma_i^2$ has no direct effect on the election result but is useful for strategic analysis.

    In this strategy, the only information voters have on the candidates
    is the polling data and their own personal associated utilities.
    To model this, we will assume a voter's belief of the distribution
    of a candidate $c_i$'s score $\mu_i$ follows a normal distribution
    with the same variance as would be expected from the sample distribution,
    which by the central limit theorem is $\frac{\sigma_i^2}{P}$.
    These belief distributions have means we will call $x_1, x_2, \dots x_C$
    --- the numerical values of these are not important,
    but they are known to be in the polling order,
    as shown in this plot of the believed distributions of $\mu_4$, $\mu_3$, $\mu_2$, and $\mu_1$:

    \begin{tikzpicture}
        \begin{axis}
            [
            no markers, domain=0:1, samples=100,
            axis x line=bottom,
            axis y line=none,
            xlabel=score,
            every axis x label/.style={at=(current axis.right of origin),anchor=west},
            height=5cm, width=12cm,
            xticklabels={$x_4$, $x_3$, $x_2$, $x_1$}, xtick={0.2, 0.4, 0.55, 0.65}, ytick=\empty,
            enlargelimits=false,
            grid = major
            ]
            \addplot [very thick,cyan!50!black] {gauss(0.2,0.05)};
            \addplot [very thick,cyan!50!black] {gauss(0.4,0.05)};
            \addplot [very thick,cyan!50!black] {gauss(0.55,0.05)};
            \addplot [very thick,cyan!50!black] {gauss(0.65,0.05)};
        \end{axis}
    \end{tikzpicture}

    Since the $P$ is large, we can assume the variance $\frac{\sigma_i^2}{P}$ is very small:

    \begin{tikzpicture}
        \begin{axis}
            [
            no markers, domain=0:1, samples=300,
            axis x line=bottom,
            axis y line=none,
            xlabel=score,
            every axis x label/.style={at=(current axis.right of origin),anchor=west},
            height=5cm, width=12cm,
            xticklabels={$x_4$, $x_3$, $x_2$, $x_1$}, xtick={0.2, 0.4, 0.55, 0.65}, ytick=\empty,
            enlargelimits=false,
            grid = major
            ]
            \addplot [very thick,cyan!50!black] {gauss(0.2,0.01)};
            \addplot [very thick,cyan!50!black] {gauss(0.4,0.01)};
            \addplot [very thick,cyan!50!black] {gauss(0.55,0.01)};
            \addplot [very thick,cyan!50!black] {gauss(0.65,0.01)};
        \end{axis}
    \end{tikzpicture}

    The voter's vote can only change the outcome of the election if the
    values of the actual two highest scores are a ``near tie'', within $\frac{1}{V}$ of each other.
    Otherwise, the vote cannot change the outcome of the election
    --- this case can be considered a fixed component of the expected utility,
    and so can be ignored for reasoning about the optimal vote.
    The following reasoning will assume there is some near tie.

    Consider some fixed $\mu_1$ near $x_1$.
    This can be within, say, $1000$ standard deviations.
    With large $P$, the standard deviation is small enough that this is nowhere close to $x_2$.
    Since the probability that $\mu_1$ is near $x_1$ is large,
    its contribution dominates the expected utility,
    and we can ignore the case that $\mu_1$ is not near $x_1$.
    Likewise, the probability that every other candidate $c_i$ has $\mu_i \le \mu_1 + \frac{1}{V}$ is large,
    and we can ignore the other case.
    Now, the only result-determining near ties that can happen are between $c_1$ and some other $c_i$.
    The chance of such a near tie is proportional to $\frac{1}{V} \exp\left(-\frac{P(x_1 - x_i)^2}{2\sigma^2}\right)$.
    Since $P$ is large,
    this chance is much larger for $c_2$ than for any other $c_i$.
    Therefore, the voter must optimize the vote for the case that $c_1$ and $c_2$ have a near tie,
    and so will choose some vote that maximizes or minimizes the 1st and 2nd components according to
    whether the voter prefers $c_1$ over $c_2$.
    This determines the vote subsets $X_1$ and $X_2$.

    After narrowing down the potential votes to $X_2$ by considering when $c_1$ is a member of the near tie,
    there may still be many possible votes to make.
    It is already a very small chance for the optimization to $X_2$ to affect the outcome of the election,
    and it is even less likely for any remaining optimization to do so
    --- and so the restriction to $X_2$ is already ``good enough'' for most purposes
    (which will be useful for non-COAF systems).
    However, further optimizations can still be determined using similar reasoning.

    % TODO: complete this proof, or just cut it off here.

    \subsubsection{Non-COAF Systems}

    Properties of non-COAF systems, and description of voter behavior.

    \subsection{A List of Voting Systems Considered}\label{subsec:a-list-of-voting-systems-considered}

    A list of voting systems we simulate.


    \section{Utility Distributions}\label{sec:utility-distributions}

    Discussion of utility distributions, including uniform vs.\ normal
    vs.\ multimodal, and random vs.\ issue-based.


    \section{Simulation}\label{sec:simulation}

    Discussion of implementation of simulation.


    \section{Analysis}\label{sec:analysis}

    Analysis of some plots.


    \section{Conclusions}\label{sec:conclusions}

    Conclusions of the project.


    \section{Appendix A: Plots}\label{sec:appendix-a:-plots}

    All the plot images.


    \bibliographystyle{plain}
    \bibliography{paper}{}

\end{document}