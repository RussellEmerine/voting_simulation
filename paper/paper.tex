% Preamble
\documentclass[11pt]{article}

% Packages
\usepackage[letterpaper,margin=1in]{geometry}
\usepackage{amsmath}
\usepackage{amsthm}
\usepackage{amsfonts}
\usepackage{amssymb}
\usepackage{fancyhdr}
\usepackage{mathtools}
\usepackage{tikz-cd}
\usepackage{stmaryrd}
\usepackage{enumerate}
\usepackage{capt-of}
\usepackage{xurl}
\usepackage{hyperref}

% Document
\begin{document}
    \setlength{\parindent}{0pt}
    \setlength{\parskip}{5pt}
    \setlength{\headheight}{14.49999pt}
    \addtolength{\topmargin}{-1.59999pt}

    \title{Simulating Voting Systems}
    \author{\normalsize{Russell Emerine (\url{remerine@andrew.cmu.edu}), Daniel Sleator (\url{sleator@andrew.cmu.edu})}}
    \date{}

    \maketitle

    This project's source code is available online at \url{https://github.com/RussellEmerine/voting_simulation}.


    \section{Abstract}\label{sec:abstract}

    \textbf{
        Much of the past work on voting systems focuses on ranked voting systems,
        which have a number of limitations such as Arrow's Theorem \cite{Arrow}.
        Furthermore, evaluation of their performance becomes more difficult when considering strategic voting behavior.
        In this paper we consider ranked voting systems as well as the more general class of rated voting systems.
        Like Smith, we evaluate ranked and rated voting systems
        under a Monte Carlo simulation model of voter utilities and behaviors \cite{Smith}.
        We replicate Smith's results with a wider selection of voting systems and voter utility distributions
        and conclude that range voting has the best performance.
    }


    \section{Background}\label{sec:background}

    \textcolor{red}{TODO: include?}


    \section{Voting Systems}\label{sec:voting-systems}

    A \textit{voting system} is a system where
    a (usually large) number of \textit{voters} cast \textit{ballots} of some kind,
    and the ballots are used to select one of
    a (usually small, but greater than one) number of \textit{candidates}.

    Some formulations allow the output to be a set of candidates or an ordering of candidates.
    For simplicity, we will only consider systems that output a single winner.
    We will also focus on cases with three or more candidates, as with two candidates,
    almost all voting systems we consider exhibit the same optimal behavior.

    We first describe several types of voting systems,
    then list and describe the particular voting systems we consider in our simulation.

    \subsection{``Obviously Bad'' Voting Systems}\label{subsec:obviously-bad-voting-systems}

    Some voting systems that we consider, such as Random Winner and Worst Candidate,
    are useful only as a frame of reference.

    \subsection{Ranked Voting Systems}\label{subsec:ranked-voting-systems}

    In a ranked voting system, ballot information can be encoded into an ordering of the candidates.
    For instance, plurality voting only uses the most favored candidate from each ballot,
    while Borda voting uses the whole ordering to produce the weighted sum.
    Ranked voting systems traditionally output an ordering of candidates;
    to choose our single winner we can simply choose the most favored candidate in the outcome.

    Arrow's theorem states that, if there are at least three candidates,
    ranked voting systems, no voting system may satisfy all of the following properties \cite{Arrow}:

    \begin{itemize}
        \item Pareto efficiency:
        If every voter places $A$ before $B$, then the outcome also places $A$ before $B$.
        (This can be weakened to non-imposition,
        i.e.\ that for any $A$ and $B$, there is some cast of votes such that
        the output places $A$ before $B$ \cite{Wilson}.)
        \item Non-dictatorship:
        There is no voter whose ordering is always the same as the outcome.
        \item Independence of irrelevant alternatives:
        In two elections with the same number of voters where each pair of
        corresponding voters has the same relative ordering of $A$ and $B$,
        the outcomes of both elections have the same relative ordering of $A$ and $B$.
    \end{itemize}

    The fact that it is impossible to have all three of these very reasonable conditions is a
    severe limitation of ranked voting systems.
    However, not all voting systems are ranked voting systems.

    \subsection{Rated Voting Systems}\label{subsec:rated-voting-systems}

    A rated voting system is a voting system whose ballot information allows voters to express
    how strongly they support candidates.
    Since these are not ranked voting systems, Arrow's theorem does not apply,
    and it is completely possible to have Pareto efficiency,
    non-dictatorship, and independence of irrelevant alternatives.

    However, they are still subject to some restrictions, notably Gibbard's theorem,
    that no deterministic process of collective decision may satisfy all of the following properties \cite{Gibbard}:

    \begin{itemize}
        \item The process has more than two possible outcomes.
        \item There is no voter who singlehandedly determines the outcome.
        \item The game-theoretically optimal ballot for a voter will not depend on
        the voter's beliefs on other voters' ballots.
    \end{itemize}

    When applied to voting systems, this implies that non-dictatorial voting systems
    with three or more possible outcomes (ranked, rated, or otherwise)
    require some kind of strategic voting.

    \subsubsection{An Aside about Determinism}

    Most voting systems do not have a good way of handling ties without
    using randomness or allowing multiple winners.
    However, when the number of voters is reasonably large, the chance of a tie is negligible.
    We will consider voting processes that are deterministic when there are no ties ``good enough.''

    \subsection{COAF Systems}\label{subsec:coaf-systems}

    COAF specification, and proof of optimal voter behavior.

    \subsection{Non-COAF Systems}\label{subsec:non-coaf-systems}

    List of non-COAF systems, and description of voter behavior.


    \section{Utility Distributions}\label{sec:utility-distributions}

    Discussion of utility distributions, including uniform vs.\ normal
    vs.\ multimodal, and random vs.\ issue-based.


    \section{Simulation}\label{sec:simulation}

    Discussion of implementation of simulation.


    \section{Analysis}\label{sec:analysis}

    Analysis of some plots.


    \section{Conclusions}\label{sec:conclusions}

    Conclusions of the project.


    \section{Appendix A: Plots}\label{sec:appendix-a:-plots}

    All the plot images.


    \bibliographystyle{plain}
    \bibliography{paper}{}

\end{document}